\begin{quotation}
``\emph{不是Balard发现了溴,而是溴发现了Balard。}''

{\ttfamily ``Balard did not discover bromine, rather bromine \\
discovered Balard.''
\\--- Justus von Liebig}

``\emph{我们把有机化学定义为碳的化合物的化学。}''

{\ttfamily ``We define organic chemistry as the chemistry of \\
carbon compounds.''
\\ --- August Kekule}

{\ttfamily ``Chem is try''
\\ --- Anonymous}
\end{quotation}

\section{怎么表示有机化合物?}
当我们想谈论关于一个有机化合物时,我们要让别人明白我们指的是哪个,于是表示有机化合物的方式就很重
要,它们就像名字一样。不过很多方式除了有名字的作用之外,还能直观的展示一些它们的特点。

\subsection{用式子表示有机化合物}
常用的几种用式子表示有机化合物的方法有:
\begin{enumerate}
\item 名称
\item 化学式
\item 结构式
\item 结构简式
\item 键线式
\end{enumerate}

\paragraph{名称}
名称没什么好说的,比如甲烷,乙酸,乙二酸,硝基苯。名称对应的化合物一般是唯一的,但是一个化合物却不一
定只有一个名称。比如氢氧化钠(\chemfig{NaOH})又叫烧碱,乙酸(\chemfig{CH_3COOH})又叫醋酸。

\paragraph{化学式}
化学式展示了组成该化合物的所有原子及其数目(或比例),但无法体现化学键。只要按一定的顺序写出各个原
子,并把它们的数目标在右下角就是化学式中的分子式,比如\chemfig{C_6H_6}, \chemfig{C_2H_6O}。
写比例(最简整数比)的是实验式,在某些问题上很有用。可以有多种的化合物的化学式相同,这导致了同分异构
体的出现,我们会在后来简要提及它。

\paragraph{结构式}
结构式是用短线(\chemfig{-})等表示化学键,连接起元素符号代表的元素的一种化学组成式。它能表示出化
合物的化学键,元素组成,以及一定的空间结构。几个例子:
\begin{center}
甲烷: \chemfig{H-C(-[2]H)(-[6]H)-H},
乙酸: \chemfig{H-C(-[2]H)(-[6]H)-C(=[2]O)-O-H}。
\end{center}

\paragraph{结构简式}
在结构式中的碳氢键常常很多,比较麻烦。于是人们常常使用结构简式 --- 省去了碳氢键\footnote{有
时也省去氧氢键等常用键。}的结构式来表示有机化合物。比如1-丙醇可被简化为
\begin{center}
\chemfig{OH-CH_2-CH_2-CH_3}
\end{center}

\paragraph{键线式}
键线式进一步简化了结构式,把碳骨架上C, H的元素符号也省去了。注意书写键线式时只能省略碳骨架上的H。
例子:
\begin{center}
正己烷:\chemfig{-[:30]-[:-30]-[:30]-[:-30]-[:30]-[:-30]}\\
3-戊烯酸:\chemfig{-[:30]=[:-30]-[:30]-[:-30](=[6]O)-[:30]OH}
\end{center}

在最后,我们想说,无论怎么表示,最终的目的是要简洁有效地让别人明白你想表达的是什么。即使是一种
化合物,在不同的上下文下,也有很多不同的合适的表达方法。另外,不要拘泥于规则,如果有一种比已有规则
更易懂,也不会让人产生混淆的表达方法,用它也没有什么关系。

\subsection{用模型表示有机化合物}
模型能很好地展示有机化合物的各原子的相对大小,空间组成。常用的模型有球棍模型,填充模型等。
球棍模型把原子做成小球,用棍子按空间组成插在一起。可以从中看出空间结构,原子相对大小,但是难以看出
相对距离。填充模型是把整个分子按比例放大,其中的原子近似为球状,比较容易看出相对距离和分子相对大
小,但是空间结构比较难看出来(相对球棍模型)。


\section{链烃(脂肪烃)的性质}
链烃中的C原子不连成环。比如甲烷(\chemfig{CH4}),乙烯(\chemfig{CH2=CH2}),\\
丙炔(\chemfig{CH~C-CH3})等都是链烃。

虽然这一节是讲链烃,但我们也同时提一下环烃,它们的C原子连成环。而环烃还可细分成芳香烃和脂环烃。
\textbf{芳香烃是含有苯环的环烃},
\begin{center}
苯: \chemfig{*6(-=-=-=)} 或 \chemfig{**6(------)}
\end{center}
本身也属于它。前者是苯的凯库勒式,凯库勒认为苯环是单双键交替的环。虽然现代化学已经证明了苯不是这样
的,它的C之间有更复杂的化学键,但凯库勒富有创造力和想象力的凯库勒式极大地推进了该领域的发展。现在
认为苯的C由单键和跨越整个环的大$\pi$键相连。

\subsection{链烃的结构特点}
\textbf{烷烃是C原子间全部以饱和键(单键)相连的链烃}\emph{或甲烷}。

% 'n' in math mode cannot be written into right footers in chemfig, 
% so we use 'n' without '$'.
当C原子数n大于等于3时,中间的每个C要与2个C相连,每个C上只能有2个H,而两边的C上有3个H,故此时
烷烃的化学式为\chemfig{C_{n}H_{2n+2}}。

很易证明,当n=1,n=2时这化学式仍然成立。故烷烃的化学通式为\\
\chemfig{C_{n}H_{2n+2}}。

\textbf{烯烃是C原子间有双键的链烃}。

很易看出,单烯烃的通式为\chemfig{C_{n}H_{2n}}。

\textbf{炔烃是C原子间有三键的链烃}。

很易看出,单炔烃的通式为\chemfig{C_{n}H_{2n-2}}。

\subsection{链烃的性质}
由于不饱和键的存在,烯烃,炔烃与烷烃的化学性质有着较大的差别。前两者能发生很多后者不能的反应,比如
加成反应等。而很多东西都能与烯烃,炔烃加成,比如卤素单质,水等。这使得用类似溴水的东西区分它们成为
可能。另外,酸性高锰酸钾能够氧化这样的不饱和键,所以也可以用酸性高锰酸钾来区分它们。

它们的化学性质区别还体现在很多不同的方面。甚至不止是化学性质,它们的一些物理性质(如熔沸点)也有显
著的不同。表\ref{table:attributesOfHydrocarbon} 展示了烷烃和烯烃,炔烃的主要化学性质区别。

\subsubsection{链烃的化学性质}
\begin{table}[!hbpt]
\begin{center}
\begin{tabular}{p{2cm}|p{2cm}|p{3cm}|p{3cm}|}
 & 烷烃 & 烯烃 & 炔烃 \\
\hline
活动性 & 较稳定 & 比烷烃活泼 & 较活泼 \\
\hline
取代反应 & 能够与卤素取代 & \multicolumn{2}{c|}{高中不涉及} \\
\hline 
加成反应 & 不能发生 & 
\multicolumn{2}{p{5cm}|}{能与\chemfig{H_2}, \chemfig{X_2}, \chemfig{HX}, 
\chemfig{H_2O}, \chemfig{HCN}等加成} \\ \hline
燃烧 & 产生淡蓝色火焰 & 火焰明亮,有黑烟 & 火焰明亮,有浓烟\\
\hline
与酸性高锰酸钾 & 不反应 & \multicolumn{2}{c|}{使它退色} \\
\hline
加聚反应 & 不能发生 & \multicolumn{2}{c|}{能发生} \\
\hline
鉴别 & \multicolumn{3}{|c|}{可用溴水,酸性高锰酸钾鉴别} \\
\hline
\end{tabular}
\end{center}
\caption{烷烃和烯烃,炔烃的主要化学性质区别}
\label{table:attributesOfHydrocarbon}
\end{table}

\subsubsection{链烃的物理性质}
链烃的物理性质随C原子数n的增加呈现规律性变化。
n小于5时,为气态,之后随它增加过渡到液态和固态。
随n的增加\textbf{沸点}增加;\textbf{相对密度}增大。
同分异构体中,支链数目越多,\textbf{沸点}越低。
都难溶于水,易溶于有机溶剂。

\subsection{涉及链烃的化学反应}

\subsubsection{涉及烷烃的化学反应}
由于烷烃的化学键键能很高,导致其能发生的化学反应比较少。
一般见得最多的是取代反应 (如甲烷与氯气):
\begin{center}
\schemestart 
\chemfig{CH_4} \+ \chemfig{Cl_2} 
\arrow{->[$h\nu$]}
\chemfig{CH_3Cl} \+ \chemfig{HCl} 
\schemestop
\end{center}
要注意的是这化学反应方程式中只出现了一种主要产物,这是因为
\begin{center}
\schemestart 
\chemfig{CH_3Cl} \+ \chemfig{Cl_2} 
\arrow{->[$h\nu$]}
\chemfig{CH_2Cl_2} \+ \chemfig{HCl} 
\schemestop
\end{center}
事实上该反应产生很多种副产物,所以该反应一般不被用于工业生产。

烷烃的裂解在断裂碳碳键的同时,还要断裂碳氢键生成碳碳双键。网课中给的例子是:丁烷的裂解
\begin{center}
\schemestart
\chemfig{C_4H_{10}} \arrow{->[高温]} 
\chemfig{CH_4} \+ \chemfig{CH_2=CH-CH_3}
\schemestop
\end{center}
\begin{center}
\schemestart
\chemfig{C_4H_{10}} \arrow{->[高温]} 
\chemfig{C_2H_6} \+ \chemfig{CH_2=CH_2}
\schemestop
\end{center}

点燃时,通式为
\begin{center}
\schemestart
\chemfig{C_{n}H_{2n+2}} \+ $\frac{3\text{n} + 1}{2}$\chemfig{O_2}
\arrow{->[点燃]} n\chemfig{CO_2} \+ (n + 1)\chemfig{H_2O}
\schemestop
\end{center}

\subsubsection{涉及烯烃的化学反应}
烯烃中的碳碳双键可以发生加成反应,可以和卤素,水等物质加成。
这两个反应方程式如下:
\begin{center}
\schemestart
\chemfig{=[:30]-[:-30]} \+ \chemfig{Br_2} \arrow{->}
\chemfig{Br-[2]-[:30](-[-2]Br)-[:-30]}
\schemestop
\end{center}
\begin{center}
\schemestart
\chemfig{=[:30]-[:-30]} \+ \chemfig{H_2O} \arrow{->[Cat]}
\chemfig{-[:30](-[-2]OH)-[:-30]} 或 \chemfig{HO-[:30]-[:-30]-[:30]}
\schemestop
\end{center}

它还可以发生加聚反应,如:
\begin{center}
\schemestart
n\chemfig{CH_2=CH_2} \arrow{->[引发剂]}
\chemfig{-[@{op}]CH_2-CH_2-[@{cl}]}
\polymerdelim[height=6pt,delimiters={[]}]{op}{cl}
\schemestop
\end{center}

碳碳双键与酸性高锰酸钾溶液反应生成的产物与两边的东西有关,这里暂不讨论,下面会有个表
\ref{table:CHKMnO4} 详细介绍这些东西。

\subsubsection{涉及炔烃的化学反应}
加成和加聚反应方面,炔烃与烯烃很相似,这里就不提了。
提一下与酸性高锰酸钾溶液反应,
\begin{center}
\schemestart
\chemfig{CH~CH} \arrow{->[\chemfig{KMnO_4}]} \chemfig{CO_2}
\schemestop
\end{center}

\begin{table}[!hbtp]
\begin{center}
\begin{tabular}{r|c}
\hline
被氧化的部分 & 氧化产物 \\
\hline
\chemfig{CH_2=} & \chemfig{CO_2}, \chemfig{H_2O} \\
\hline
\chemfig{RCH=} & \chemfig{R-C(=[-2]O)-OH} \\
\hline
\chemfig{C(-[3]R')(-[-3]R'')=} & \chemfig{C(-[3]R')(-[-3]R'')=O} \\
\hline
\chemfig{HC~} & \chemfig{CO_2} \\
\hline
\chemfig{R-C~} & \chemfig{R-COOH} \\
\hline
\end{tabular}
\caption{炔烃和烯烃与酸性高锰酸钾溶液反应}
\label{table:CHKMnO4}
\end{center}
\end{table}

\subsection{一些例题}
下面我们来看一些关于链烃的例题。

(1) 某种气态混合的烷烃和炔烃2L,完全燃烧后生成同温同压下\\
\chemfig{CO_2} 2.8L,\chemfig{H_2O}(g) 3.2L,该混合烃是()
\begin{multicols}{2} % Two enumerate items in one line.
\begin{enumerate}[label=(\Alph*)]
\item \chemfig{CH_4}, \chemfig{C_2H_2}
\item \chemfig{C_2H_6}, \chemfig{C_2H_2}
\item \chemfig{C_3H_8}, \chemfig{C_3H_4}
\item \chemfig{CH_4}, \chemfig{C_3H_4}
\end{enumerate}
\end{multicols}

我并不喜欢做选择题的时候看选项,我认为这是取巧的行为,有的时候,这种行为使得题目失去了它的本质。
然而会有这种情况,不看选项做不出来,在这种情况下,我尽量从选项中获取最少信息。我们来看这道题,
在不知道炔烃的三键数的时候,是无法被做出来的。所以我们只能看一眼选项,然后发现,三键数都是1。

我们明白,在都是气体,同温同压下,可以近似认为气体的体积比就是物质的量的比。于是我们设
$m,n$分别为这烷烃和炔烃的物质的量(以1L对应的物质的量为单位),$C_m, C_n$分别为它们中的
C原子的物质的量(单位同上),所以由我们之前讨论过的它们的通式,他们中H原子的物质的量分别为
$(2C_m + 2), (2C_n - 2)$.
我们有
\begin{align}
m+n &= 2 \label{eq:chem1} \\
mC_m + nC_n &= 2.8 \label{eq:chem2} \\
m(2C_m + 2) + n(2C_n - 2) &= 6.4 \label{eq:chem3}
\end{align}
大部分情况下,方程数多于未知数个数的线性方程组中无法解出任何未知数,然而并不是所有的都是这样,比如
这个方程组。为了解它,我们把\ref{eq:chem2}$\times 2$并去减\ref{eq:chem3},就得到了另一个关于
$m,n$的方程,联立它与\ref{eq:chem1}就能解出$m,n$,之后把$m,n$代入\ref{eq:chem2}和
\ref{eq:chem3}就能得出$C_m,C_m$以至于氢原子数的关系。满足这些关系的选项就是我们要的。
至于线性方程组中方程个数和未知数个数对方程组解的影响,就不是我们现在能讨论的了,这涉及到了线性代
数。

(2) 某种炔烃经与氢气加成后得到这样的物质:
\begin{center}
\chemfig{CH_3-CH(-[2]CH_3)-CH_2-CH_2-CH(-[2]CH_3)-CH_3}
\end{center},
则该炔烃可能的结构有几种?

我们来看一下,如果从三键变成单键,则两边的C原子上都多了2个H,于是能是原来三键的地方,周围的C原子
上至少有2个H。我们看到,得到的物质中这样的地方只有一处,在中间,于是可能的结构只有一种。

化学到这里就结束了。事实上,我觉得化学这个学科,至少在目前的阶段,记笔记的需求对我来说不大。我以上
写了这么多虽然部分包含了我的一些思考,但是我写这些主要还是为了学习在\LaTeX 里编辑化学。