\begin{quotation}
``自然和自然的法则在黑夜中隐藏:上帝说,`让牛顿去吧!' 于是一切都被照亮。''

{\ttfamily
\begin{sloppypar}
``Nature and Nature's laws lay hid in night:
God said, `Let Newton be!' and all was light.''
\end{sloppypar}

--- Alexander Pope}

``物理就像性一样,它的确能带来一些有用的结果,但那不是我们为什么做它的原因。''

{\ttfamily
``Physics is like sex: sure, it may give some \\
practical results, but that's not why 
we do it.''

--- Richard Feynman}

\end{quotation}

\section{光学初步}
\subsection{光的折射定律 --- Snell定律}
先不谈网课上的内容,来看一个课本上的例子。课本上的光的折射定律大家应该很熟悉了,
\[
\frac{\sin{\theta_1}}{\sin{\theta_1}} = \frac{v_1}{v_2} = \frac{n_1}{n_2}
\]
这个定律既可以被从费马原理\cite{FMsPrinciple}推导出来,也可以从我们课本上介绍的惠更斯原理被推
导出来\footnote{事实上它还可以从麦克斯韦方程组,或是从能量守恒和动量守恒,或是从平移对称性原理等
中被推导出来。然而这些推导无法被做到在中学阶段。}。

\subsubsection{用费马原理推导Snell定律}
我们先看一下用费马原理推导它。费马原理告诉我们,光线跑的路不是别的,而是用时最短的路。现在,如图
\ref{fig:DeriveSenllsLawFromFermats},
我们在线两侧有两个不同的介质区域$A, B$,其中分别存在点$P, Q$,现在有光要从$P$跑到$Q$,我们尝试
用费马原理找到它的路径。

\begin{figure}[!hbpt]
\begin{center}
\begin{tikzpicture}
\coordinate[label=left:{$P$}] (P) at (3,2);
\coordinate[label=right:{$Q$}] (Q) at (-2,-3);
\coordinate[label=above left:{$M$}] (M) at (0,0);

\node[anchor=west] at (0,1.5) {$A$};
\node[anchor=east] at (0,-1.5) {$B$};

\draw[very thick] (-4,0) -- (4,0);
\draw[dashed] (0,2.5) -- (0,-2.5);

\draw[->] (P) -- (1.5,1);
\draw (1.5,1) -- (M);
\draw[->] (M) -- (-1,-1.5);
\draw (-1,-1.5) -- (Q);

\end{tikzpicture}
\end{center}
\caption{用费马原理推导Snell定律}
\label{fig:DeriveSenllsLawFromFermats}
\end{figure}

\begin{proof}
首先证明,光在两个介质内跑的不是别的,而是直线。首先,光无论怎么跑,都要经过分界线上的某一点M,于是
连接$PM,PN$,显然,光跑别的路径要比它跑直线跑的更长,就得出了光要跑直线。得出路径后就好算时间了。
设$P,Q$到分界线的距离分别为$a,b$,它们之间的水平距离为$l$,设$M,P$间的水平距离为$x$,光在$A,B$
中的速度分别为$v_1,v_2$,
则时间
\begin{equation}
t = \frac{\sqrt{a^2 + x^2}}{v_1} + \frac{\sqrt{b^2 + (l-x)^2}}{v_1}
\end{equation}
求导
\begin{equation}
\frac{\mathrm{d}t}{\mathrm{d}x} = \frac{x}{v_1\sqrt{a^2 + x^2}} + 
\frac{-(l-x)}{v_2\sqrt{b^2 + (l-x)^2}}
\end{equation}
很易看出,导数零点就是极小值点。
注意到,
\begin{align*}
\frac{x}{v_1\sqrt{a^2 + x^2}} &= \sin{\theta_1} \\
\frac{(l-x)}{v_2\sqrt{b^2 + (l-x)^2}} &= \sin{\theta_2}
\end{align*}
,其中$\theta_1,\theta_2$分别是$PM,QM$与法线成的角。
这就完成了证明。
\end{proof}

用惠更斯原理推导这里就不讲了。