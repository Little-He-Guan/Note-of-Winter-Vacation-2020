\begin{quotation}
``自然是不是可能像在这些原子实验里的这样荒诞?''

{\ttfamily
\begin{sloppypar}
``Can nature possibly be so absurd as it seemed to us in these atomic 
experiments?''
\end{sloppypar}

--- Werner Heisenberg}

``物理就像性一样,它的确能带来一些有用的结果,但那不是我们为什么做它的原因。''

{\ttfamily
``Physics is like sex: sure, it may give some \\
practical results, but that's not why 
we do it.''

--- Richard Feynman}

\end{quotation}

\section{光学初步}
光学主要研究光的性质,现象及其应用。人类对于光的研究很早就开始了,随着人们对光认识和其它领域研究
的进步,光学逐步发展,到了现在,人们已经不得不用量子力学来解释光的一些行为了。

我们先来谈谈在历史上被大量研究的光学,几何光学。它试图将光看作一条射线,并能成功解释光的折射,反射
等一些现象。反射就不提了,早在初中我们都已接触过并认识了这方面的内容。

\subsection{光的折射}
先不谈网课上的内容,来看一个课本上的例子。课本上的光的折射定律大家应该很熟悉了,
它又被称为Snell定律,因为Snell (Willebrord Snellius)最先给出了它的一个数学关系式。
\[
\frac{\sin{\theta_1}}{\sin{\theta_1}} = \frac{v_1}{v_2} = \frac{n_1}{n_2}
\]
这个定律既可以被从费马原理\cite{FMsPrinciple}推导出来,也可以从我们课本上介绍的惠更斯原理被推
导出来\footnote{事实上它还可以从麦克斯韦方程组,或是从能量守恒和动量守恒,或是从平移对称性原理等
中被推导出来。然而这些推导无法被做到在中学阶段。}。

\subsubsection{用费马原理推导Snell定律}
我们先看一下用费马原理推导它。费马原理告诉我们,光线跑的路不是别的,而是用时最短的路。现在,如图
\ref{fig:DeriveSenllsLawFromFermats},
我们在线两侧有两个不同的介质区域$A, B$,其中分别存在点$P, Q$,现在有光要从$P$跑到$Q$,我们尝试
用费马原理找到它的路径。

\begin{figure}[!hbpt]
\begin{center}
\begin{tikzpicture}
\coordinate[label=left:{$P$}] (P) at (3,2);
\coordinate[label=right:{$Q$}] (Q) at (-2,-3);
\coordinate[label=above left:{$M$}] (M) at (0,0);

\node[anchor=west] at (0,1.5) {$A$};
\node[anchor=east] at (0,-1.5) {$B$};

\draw[very thick] (-4,0) -- (4,0);
\draw[dashed] (0,2.5) -- (0,-2.5);

\draw[->] (P) -- (1.5,1);
\draw (1.5,1) -- (M);
\draw[->] (M) -- (-1,-1.5);
\draw (-1,-1.5) -- (Q);

\end{tikzpicture}
\end{center}
\caption{用费马原理推导Snell定律}
\label{fig:DeriveSenllsLawFromFermats}
\end{figure}

\begin{proof}
首先证明,光在两个介质内跑的不是别的,而是直线。首先,光无论怎么跑,都要经过分界线上的某一点M,于是
连接$PM,PN$,显然,光跑别的路径要比它跑直线跑的更长,就得出了光要跑直线。得出路径后就好算时间了。
设$P,Q$到分界线的距离分别为$a,b$,它们之间的水平距离为$l$,设$M,P$间的水平距离为$x$,光在$A,B$
中的速度分别为$v_1,v_2$,
则时间
\begin{equation}
t = \frac{\sqrt{a^2 + x^2}}{v_1} + \frac{\sqrt{b^2 + (l-x)^2}}{v_1}
\end{equation}
求导
\begin{equation}
\frac{\mathrm{d}t}{\mathrm{d}x} = \frac{x}{v_1\sqrt{a^2 + x^2}} + 
\frac{-(l-x)}{v_2\sqrt{b^2 + (l-x)^2}}
\end{equation}
很易看出,导数零点就是极小值点。
注意到,
\begin{align*}
\frac{x}{v_1\sqrt{a^2 + x^2}} &= \sin{\theta_1} \\
\frac{(l-x)}{v_2\sqrt{b^2 + (l-x)^2}} &= \sin{\theta_2}
\end{align*}
,其中$\theta_1,\theta_2$分别是$PM,QM$与法线成的角。
这就完成了证明。
\end{proof}

用惠更斯原理推导这里就不讲了。

\subsubsection{折射率和全反射}
显然,当两边介质一样的时候,光在里面传播的速率也一样,那么这个折射角的正弦之比就是定值。我们把这个
比值记为$n$. 如果令$\theta_1$表示光在真空中的入射角,这时的$n$就被定义为某种介质的(绝对)折射
率。光的相对折射率就很好理解了,就是把一种介质的入射角作为$\theta_1$,这样$n$就成为了某介质相对
于该介质的折射率。

从上面的推导我们能看出,折射率也可用$\frac{v_1}{v_2}$即是速度的比值表示。因为光在空气中传播的速
度和在真空中传播的速度几乎相等,我们就可以认为,某种介质的绝对折射率就近似等于以空气中入射角为
$\theta_1$的$n$. 

我们可能会发现一个问题,首先,光在真空中跑得最快,所以$n$总是大于1的。而因为$n$总是大于1的,
$\theta_1,\theta_2$就不能跑遍$(0,\frac{\pi}{2})$之间的每一个值。准确地讲,因为
\begin{equation}
\sin{\theta_1} = n\sin{\theta_2}
\end{equation}
所以$\theta_2$只能在$(0,\arcsin{\frac{1}{n}})$里被取值。一旦它越出这个范围,$\theta_1$就成
为了未被定义的。为了研究这种情况下的折射是个什么情况,我们还需要重新做实验来看看。

这时做实验我们显然就不能通过变动$\theta_1$来做了,而是要变动$\theta_2$使它越出
$(0,\arcsin{\frac{1}{n}})$的范围。大量的实验表明,在这种情况下,折射完全丢失了,只剩下反射光。
我们把这种现象称为全反射。

这样,我们把发生全反射的临界角记作$C$,而
\begin{equation}
C = \arcsin{\frac{1}{n}}
\end{equation}

\subsubsection{光的色散和它与折射的关系}
据说是牛顿最先发现了这一现象---他把阳光透过三棱镜后,发现它变成了一条七种颜色的色带。

我们尝试用我们刚谈过的折射理论来研究这个问题。把阳光看成一道光从空气入射进入三棱镜(玻璃),然后再
出射,而最后出射的角度不一样。通过分析表明,这意味着折射率的不同。然而一束光怎么会有不同的折射率在
相同的介质里呢?人们经过探索和思考,认为这是因为阳光是多种色光的结合,而各种色光在玻璃的折射率不一
样。根据我们之前的公式,这暗示着不同颜色的光在相同介质中的传播速度可能不同。

在之后,我们会谈到光的波动性,事实上,光的波动性最先被提出不是因为别的,而是为了解释颜色的起源。
科学家胡克(Robert Hooke)开发了一个振动理论为了解释颜色的起源,并且他把光和水波的传播做了比较。
之后,惠更斯(Christiaan Huygens)做出了一个光的波动说的数学理论。

\subsection{光的波动性}
在人们研究光的过程中,发现了一些无法用几何光学来解释的事实。我们之前提过,惠更斯等人提出了波动说,
而与此对应,牛顿提出了微粒说,认为光是由一些特殊性质的微粒构成的。由于牛顿的权威性,在长达一个多世
纪的时间内,波动说一直都处于弱势,直到物理学家托马斯(Thomas Young)用实验证实了光具有波的重要性
质---著名的双缝干涉实验。

\subsubsection{光的干涉}
在课本上,给出了一种推导两个亮/暗条纹之间距离$\Delta r$与
与双缝间距离$d$、挡板与屏距离$l$和光波长$\lambda$的近似公式
\begin{equation}
\Delta r = \frac{l}{d}\lambda
\end{equation}
该公式当$l$远大于$d$时近似成立。

我们知道,双缝干涉的原理是当屏上一点到两缝的距离差刚好是
光波长的整数倍时,两个相干波源发出的某个波峰就会同时到达那一点,叠加加强。事实上,不仅是双缝,只要是
能产生相干波源(即是,两个震动情况总是相同)的一些器械也能在某种限定下使我们观察到干涉现象。

另外,不仅仅是这样的相干波源,即使是跟某一个波源的震动相差一些相位的另一个波源,它们之间也能发生
干涉。比如薄膜干涉,来自前表面的反射光和来自后表面的反射光振动相差一些相位,如果它们到人眼的距离差
正好等于它们差的相位乘波长的话,就得到了加强光。

\subsubsection{光的衍射}
最早仔细观察衍射并对其命名的是意大利物理学家弗朗西斯科\\
(Francesco Maria Grimaldi)。他指出:
\begin{quote}
``光不仅会沿直线传播、折射和反射,还能够以第四种方式传播,即通过衍射的形式传播。''\\
``Lumen propagatur seu diffunditur non solum directe, refracte, ac 
reflexe, sed etiam alio quodam quarto modo, diffracte.''
\end{quote}
并把衍射命名为\emph{diffraction},从拉丁文的\emph{diffringere}里得到启发,意思是
``破碎成碎片'',来表示光的传播方向被打碎成不同的各种方向。

衍射是指的是波穿过障碍物后发生不同程度的弯曲传播。我们知道,日常生活中的光基本都是沿直线传播,
然而我们尝试用很小的孔让光通过,就会看到明显的衍射现象。

大量的实验和研究表明,光的波长越大,障碍物越小,衍射现象越明显。

\subsubsection{干涉与衍射的关系}
课本上对这一点没有提。然而我觉得这不太好,我提一下。

事实上,在干涉实验和衍射实验中,都是既有干涉又有衍射。根据惠更斯原理,如果我们把小孔上的每一点
都作为波源的话,那么衍射现象是不是也可以被认为是无限多的这些波源干涉得到的呢?

遗憾的是,对于衍射和干涉现象,现在还没有人能给出满意的解答他们间的关系是什么样的,
美国物理学家,诺贝尔物理学奖得主费曼\\
 (Richard P. Feynmann)提出:
\begin{quote}
``没有人能够令人满意地定义干涉和衍射的区别。这只是术语用途的问题,其实二者在物理上并没有什么特别
的、重要的区别。''
\end{quote}

\subsubsection{光的偏振}
既然我们已经证实了光的波动性,那么就会比较自然的想到光是横波还是纵波。这个问题课本上展示了用
偏振片做的实验,并得出了光是横波的结论。另外,还得出了自然光的振动方向是垂直于传播方向的一切
方向。

并且,它给出了定义:在垂直于传播方向的平面内只沿一个方向传播的光是偏振光。余下的课本上的知识就没有
什么好说的了。

\subsubsection{光的电磁说}
我们已经得到了这么多光的波动性质,然而我们还有一个问题没有解决,光到底是什么波?它和我们最经常
见的机械波比如声波不同,它甚至能在真空中传播。并且它传播的是这样的快,以至于在一段时间内,不少
物理学家认为光速是无穷大。

对于光是什么波的一个重要里程碑式的突破性研究,就要从麦克斯韦说起。

物理学家麦克斯韦(James C. Maxwell)从1855年就开始研究电学和磁学,然而他当时并没有想到他在这
方面的研究将揭示一个了不起的事实。他提出了法拉第(Michael Faraday)在电磁学领域的成果的数学模型。
后来,他将这些模型归纳得到了由20个方程组成的方程组,同时,他根据法拉第的力线提出了电磁场的模型。

之后,在他根据已有实验数据对电场的速度进行计算时,他发现计算出的速度和光速相差无几。他并不相信这是
一个巧合,并继续对该问题进行研究,并提出了电磁波方程,这方程从理论上预言了电磁波的存在。利用当时
已经得到的实验结果,他对电磁波的传播速度进行理论计算,发现这速度几乎就是光速。于是他在他发表的论文
《电磁场的动力学理论》中写道:
\begin{quote}
``这些结果的一致性似乎表明,光与磁是同一种物质的两种属性,而光是按照电磁定律在电磁场中传播的电磁扰
动。''
\end{quote}

尽管麦克斯韦因为胃癌英年早逝,没有看到赫兹(Heinrich R. Hertz)对电磁波的实验证明,但是
他的理论的正确性已经得到公认,并且他对光学和电磁学的联系成为了19世纪物理最伟大的成就之一。

\subsection{光的粒子性}
光的电磁说被证明还没过多久,光的波动说就受到了一次极大的挑战。光电效应的发生是经典电磁学理论无法
解释的,人们纷纷想着怎么去完善这一理论来符合实验结果。

受到普朗克(Max Karl Ernst Ludwig Planck)的理论的启发,年轻的爱因斯坦(Albert Einstein)
大胆地提出了光子来解释光电效应。与经典理论的最大区别是,他将光描述为一群离散的粒子,而不是连续的波
动。

课本上对于这个的介绍已经足够了,我们这里就不再写一遍了。

\subsection{光的波粒二象性}
在爱因斯坦的理论被密立根(Robert A. Millikan)做实验证实后,物理学家们接受了光既有波动性,又有粒
子性的事实。于是,那么光到底是波还是粒子呢,有人就用这个问题问过爱因斯坦。他反问``为什么光不能
既是波,又是粒子呢?''后来,大量的事实表明,光既是一种波又是一种粒子---光具有波粒二象性。

光既是一种波也是一种粒子,意思是它可以表现出波的性质,也可以表现出粒子的性质。波动性没有否认粒子
性,反之也是这样。事实上,由于光的波粒二象性,光电效应可以完全被用波动说(修正后的)解决而不需要光子
的概念,1969年兰姆(Willis Lamb, Junior)与斯考立(Marlan O. Scully)证明了这一点。

\section{量子力学初步}
量子论的观点最早是在研究黑体辐射的时候被提出来的。物理学家普朗克为了解决黑体辐射的问题,创造性地
提出能量不是被连续地而是被离散地吸收和辐射的。他提出的普朗克公式和实验数据几乎完全吻合。

量子力学还有一个重要的历史事件是光电效应,我们已经在之前提过了。

在原子领域,卢瑟福(Ernest Rutherford)的原子模型曾被公认为是正确的,然而它却有两个无法被解决的
问题,这两个问题在课本上都有,我们这里就不详细叙述了。为了解决这些问题,玻尔提出了他的H原子结构模
型,这模型引入了量子化的概念,认为电子只能在某些分立的轨道运动,这些轨道对应着一个特定的能量值。
当电子要从一个轨道跑到另一个轨道时,就要(根据两轨道的能量高低),吸收或放出能量,这过程被成为
跃迁。跃迁吸收或放出的能量以光子的形式传递,其频率为
\begin{equation}
\nu = \frac{\Delta E}{h}
\end{equation},
其中$\Delta E$是两轨道的能量差的绝对值,$h$是普朗克常量。

不过玻尔理论只能解释H原子的性质,对于别的原子就不行了。而不幸的是,即使是现在改良过的玻尔理论,
也只能解决类似H原子的原子,对别的原子仍然不能很好的解释它们的性质。

量子力学另一个重要的里程碑是德布罗意(Louis de Broglie)提出的物质波假说。
他在博士论文中声称,所有物质都拥有这类波动特性。他将物质的波长$\lambda_\text{Broglie}$和动量
$p$联系了起来:
\begin{equation}
\lambda_\text{Broglie} = \frac{h}{p}
\end{equation}

如果说爱因斯坦的光电方程把普朗克的量子理论从实物与电磁辐射之间的关系拓展到了一个基本物理特性的
话,那么德布罗意就是假设了这个物理特性对所有实物都适用,这不得不说是一个非常大胆的猜想。

而量子力学与经典理论的区别除了连续和离散之外,还有很多,其中一个比较主要的区别是:经典力学告诉我们
能准确地预言接下来会发生什么根据已有的事实,比如准确地知道一个炮弹在几秒的速度、位置等;然而在量子
力学中,却不是这样的。

一个著名的原理是不确定性原理,它指出,越能确定粒子的位置,就越不能确定它的动量。它们之间的联系
由不等式
\begin{equation}
\Delta x\Delta p \geq \frac{h}{2}
\end{equation}
来限制,其中$\Delta x,\Delta p,h$分别是位置的不确定性,动量的不确定性和普朗克常量。

寒假物理到这里就结束了。总体来说,寒假复习的这部分内容很多都是理解性的,对计算要求不大。