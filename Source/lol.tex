\newcommand{\mLoG}{League of Graphs}

\begin{quote}
{\em 欢迎来到英雄联盟\footnote{看起来应该翻译成``欢迎来到召唤师峡谷。'',但国区翻译如此。}。} \\
{\ttfamily Welcome To Summoner's Rift.}
\end{quote}

\section{为什么写这部分}
我一直都这么认为,学习和游戏并不冲突。在游戏里也能学习,同时也可以在学习里包含游戏。我平时玩过一些游
戏,里面也有很多知识可以总结。其中玩法多样且受众最广的,应该就是英雄联盟了。作为一种MOBA游戏,它已经
度过了9个年头。我曾经错误地预判它的生命期将很快结束,然而我后来发现并不是这样,因为在这个领域还没有
出现能够碾压它的对手,可能也不会出现,直到一种新的游戏类型的出现。

我个人的技术比较菜,所以本部分不是来讲我的技术心得的。然而我确信
我了解的这个网站\mLoG 在国服没有太多人知道,而同时它又功能强大很好用,于是本部分是来讲
如何使用这个网站来帮助大家提高英雄联盟技术和思路。

\subsection{什么是\mLoG ?}
\mLoG 是一个全世界知名的(应该是最强的)英雄联盟游戏数据分析以及信息收集网站,提供的服务包含且不限于
英雄分析,世界范围内回放检索/下载,排位数据分析,版本信息图表,LCS(英雄联盟锦标赛)信息等。服务是免费
的。并且,在一个大家都很担心的问题上,是不用翻墙访问的。

\section{网站整体架构}
图 展示了此网站主页的一张截图。