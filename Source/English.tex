\begin{quotation}
{\ttfamily ``We tell ourselves stories in order to live...

\begin{sloppypar}
We look for the sermon in the suicide, for the social or moral lesson in the 
murder of five. We interpret what we see, select the most workable of the 
multiple choices. We live entirely, especially if we are writers, by the 
imposition of narrative line upon disparate images, by the "ideas" with which we 
have learned to freeze the shifting phantasmagoria which is our actual 
experience.''
\end{sloppypar}
--- Joan Didion\cite{TellStoriesToLive}

``The purpose of a storyteller is not to tell \\
you how to think, but to give you 
questions to think \\
upon.''

--- Brandon Sanderson}
\end{quotation}

\newcommand{\GK}{GaoKao}
\newcommand{\DHXX}{DuHouXuXie}
\newcommand{\TriActStruct}{Three-act structure}

Writing has become more important in GaoKao recently as \DHXX{} has been added.
\DHXX{} is essentially to continue small story, so it has the qualities of a 
story. On the other hand, it is so small many aspects are restricted. In this 
part, 
we will focus on the features of \DHXX , and introduce some skills as well.

\section{Features of \DHXX}
A typical story includes several main characters, as well as their relationships,
characteristics and the world they live in. These components are usually 
introduced at the very beginning of a story. In \DHXX , the beginning is already 
given.

The \DHXX{} is rather short. We need to finish the story in about 200 words and 
this could be a challange to the development of the story. We have no time and 
space to comprehensively describe everything but to concentrate on the story and 
characters. 

As for the type of the story, things that are considered not suitable for 
students by the Chinese society, for example, romantic love, are not recommended 
to be conveyed in the story, despite the fact that most of them are highly 
effective at drawing readers attentions and are of the types of the most famous 
novels.\footnote{Affections to these things are natural to human beings, and I 
have always wonder why there is merely a country like China that uses the word 
``ZaoLian'' to show disapproval to romantic love of adolescents.}

\section{How to Write a Good Story}

\subsection{Model and Structure}
We can decide what structure to use in our story but the effect may not be good. 
It is better to use some model that is already successful and effective to struct 
our story, because we don't have much time to pratice our own model. 

\subsubsection{The \TriActStruct{} Model}
There is a classic model called the ``\TriActStruct '' that is used in narrative 
fiction.\footnote{We will only introduce its basic concepts, for more 
information, see\cite{TriActStruct}}

The \TriActStruct{} divides a story into three parts (acts) --- the Setup, the 
Confrontation, and the Resolution. The Setup establishes main characters, 
introduces their relationships and presents the world they live in. Then somehow 
an incident called the first turning point occurs, and the attemps to deal with 
the incident lead to the first plot point. Contents before the first plot point 
are often included in the given story, so what we need to do on the Setup part is 
to give our own first plot point. The first plot point generally does three 
things:
\begin{enumerate}
\item signals the end of the first act,
\item ensures life will never be the same again for the protagonist and
\item raises a dramatic question that will be answered in the climax of the 
story.
\end{enumerate}

Then the confrontation begins and our story goes around the problem initiated by 
the first turning point. The protagonists try to resolve the problem, and the 
following problems are described. Characters are developed in this act, but 
note that their characteristics may be implied in the given story. Along the way 
of trying to resolve the problem, the protagonists gradually realizes who they 
are, and the attempts to resolve the problem unexpectedly changes who they are.

Eventually the problems are resolved and the climax, where the main tensions of 
the story are brought to their most intense point and the dramatic question 
answered, leaving the protagonist and other characters with a new sense of who 
they really are.

The structure of our story can be slightly different from the \TriActStruct . We 
don't need to fit in the format exactly, and we don't have to use only one 
structure.

Let's consider an example:

\begin{quotation}
{
\sffamily
It was summer, and my dad wanted to treat me to a vacation like never before. He 
decided to take me on a trip to the Wild West.

We took a plane to Albuquerque, a big city in the state of New Mexico. We reached 
Albuquerque in the late afternoon. Uncle Paul, my dad's friend, picked us up from 
the airport and drove us up to his farm in Pecos.

His wife Tina cooked us a delicious dinner and we got to know his sons Ryan and 
Kyle. My dad and I spent the night in the guestroom of the farm house listening 
to the frogs and water rolling down the river nearby. Very early in the morning, 
Uncle Paul woke us up to have breakfast. ``The day starts at dawn on my farm,'' 
he said. After breakfast, I went to help Aunt Tina feed the chickens, while my 
dad went with Uncle Paul to take the sheep out to graze. \textcolor{blue}{I was 
impressed to see my dad and Uncle Paul riding horses. They looked really cool.}

\textcolor{blue}{In the afternoon, I asked Uncle Paul if I could take a horse 
ride, and he said yes, as long as my dad went with me. I wasn't going to take a 
horse ride by myself anyway. So, my dad and I put on our new cowboy hats, got on 
our horses, and headed slowly towards the mountains.} ``Don't be late for 
supper,'' Uncle Paul cried, ``and keep to the track so that you don't get lost!'' 
``OK!'' my dad cried back. After a while Uncle Paul and his farm house were out 
of sight. It was so peaceful and quiet and the colors of the brown rocks, the 
deep green pine trees, and the late afternoon sun mixed to create a magic scene. 
It looked like a beautiful woven blanket spread out upon the ground just for us. 

\paragraph{Paragraph 1} 
\textcolor{red}{Suddenly a little rabbit jumped out in front of my horse.}

\paragraph{Paragraph 2}
\textcolor{green}{We had no idea where we were and it was getting dark.}
}\\
\begin{large}
\textbf{Possible Version:}
\end{large}

{
\sffamily
\paragraph{Paragraph 1}
\textcolor{red}{Suddenly a little rabbit jumped out in front of my horse. Dad and 
I found it was so cute that we decided to chase it. After a while, we were 
completely lost in the forest.} There was nothing left in our sight but the 
trees. ``We may not be able to make it back to the farm house in time for 
supper.'' I thought to myself. \textcolor{green}{After a series of fruitless 
attempts to find a way out, we felt hungry and tired.}

\paragraph{Paragraph 2}
\textcolor{green}{We had no idea where we were and it was getting dark. We got 
stuck in the forest. And an unexpected shower added to the difficulty of us in 
finding a way home, for all the tracks we had made disappeared because of the 
rain.} \textcolor{orange}{I was almost on the edge of breaking down when my 
father said, ``Don't worry, my son. I remember there is a river near the farm 
house. Find the river and we will be back home.''} Finally, we found the river 
and got back to the house along it. Needless to say, we ate a late dinner.}
\\
--- GaoKao ZheJiang, 2019
\end{quotation}

The text colored with \textcolor{blue}{blue} is the \textcolor{blue}{first 
turning point}; the text colored with \textcolor{red}{red} is the \textcolor{red}
{first plot point}; the text colored with \textcolor{green}{green} is the 
\textcolor{green}{difficulties and problems the protagonists met while trying to 
solve the problem.}; and the text colored with \textcolor{orange}{orange} is the 
\textcolor{orange}{climax} of the story.

As we can see, there are some differences between the story and the \TriActStruct 
. For example, the main characters' live has not been changed after the climax. 
It is another example that the we don't need to fit in the format exactly. 
Furthermore, the possible version is such a tight story that there is hardly
something that develops the characters.

This successful structure ensures that \emph{provided you follow this sturcture, 
your story will be attrative}. The result is quiet stable, but some people don't 
like making stories this way. They might want to develop their own structures, 
which is also a possible way, though it can be more difficult. We won't describe 
this way in this note. Further reading can be found at \cite{StoryStructure}, 
where Steph Fraser tells us how to create a novel structure and some available 
structures.

\subsection{Conversation and Dialogue}
Conversation is of vital importance in most stories, so we use a subsection to 
talk about it. 

\subsubsection{Rules}
There are some rules that are useful when writing dialogue. Do remeber, 
\textbf{DIALOGUE MUST BE THERE FOR A REASON}\@. In other words, dialogue must be 
meaningful to our story. In fact, some pointless dialogue is good in a novel, but 
\DHXX{} is a damn small story, so make sure you don't write pointless 
conversation.

The reasons why a dialogue is there can be:
\begin{itemize}
\item it attracts readers' attention, or
\item it advances the story, or
\item it deepens characters' personalities, or
\item it supplies information.
\end{itemize}

To draw readers' attention, the most common technique is to include 
\emph{conflicts} in conversation. The conflict can be the disagreement between 
several characters, or the contradiction between reality and fantasy. By using 
conflicts in dialogue, the story becomes intenser. This is different from our real 
life, in which we may spend weeks without exciting things. Here is an example:

\begin{quotation}
In this example, we have a dialogue between Estella and Pip in Charles Dickens'
\emph{Great Expectations}\cite{GreatExpect}, where Estella was insulting poor Pip.

{\sffamily 
``Well?'' \par
``Well, miss?'' I answered, almost falling over her and checking myself. \par
``Am I pretty?'' \par
``Yes; I think you are very pretty.'' \par
``Am I insulting?'' \par
``Not so much so as you were last time,'' said I. \par
``Not so much so?'' \par
``No.'' \par
She fired when she asked the last question, and she slapped my face with such 
force as she had, when I answered it. \par
``Now?'' said she. ``You little coarse monster, what do you think of me now?''
\par ``I shall not tell you.''
}
\end{quotation}

We can see the conflict between Estella and Pip, and we might be curious about 
the reason why Estella was so mean, which was finally turned out as a consequence 
of her breeding.

We can also provide questions in dialogue, especially yes or no questions like 
``Will the boy get that girl?''; ``Will the characters survive the disease?''. 
Questions make the story more dramatic.

Multiple ways are available to drive the story forward. It might increase the 
suspense for what is coming, give some hints to what the characters want, or 
change the situation the characters are in.

The following illustration shows a good conversation that advances the story from 
\emph{Gone With The Wind}\cite{GWTW}.

\begin{quotation}
{\sffamily
``Sir,'' she said, ``you are no gentleman!''

``An apt observation,'' he answered airily. ``And, you, Miss, are no lady.'' He
seemed to find her very amusing, for he laughed softly again. ``No one can remain
a lady after saying and doing what I have just overheard. However, ladies have
seldom held any charms for me. I know what they are thinking, but they never
have the courage or lack of breeding to say what they think. And that, in time,
becomes a bore. But you, my dear Miss O'Hara, are a girl of rare spirit, very 
admirable spirit, and I take off my hat to you. I fail to understand what charms 
the elegant Mr.~Wilkes can hold for a girl of your tempestuous nature. He should 
thank God on bended knee for a girl with your---how did he put it?---`passion for 
living,' but being a poor-spirited wretch ---''

``You aren't fit to wipe his boots!'' she shouted in rage.

``And you were going to hate him all your life!'' He sank down on the sofa and
she heard him laughing.
}
\end{quotation}

The remarks Rhett gave Scarlett such as  ``\emph{But you, my dear Miss O'Hara, are 
a girl of rare spirit, very admirable spirit, and I take off my hat to you.}'' 
shed some light to what might have happened to Rhett and Scarlett.

Dialogue can also further develop characters' personalities. However, as mentioned 
many times before, we probably haven't enought space for such dialogue to occur. 
Then comes the information providing, which is useful in writing \DHXX . Some 
backgrounds will seem more natural if we put them into dialogue.

\subsubsection{Tips}

In dialogue novels, each speaker gets their own paragraph, and the paragraphs are 
indented, which makes the dialogue clear and brief. In \DHXX , however, we don't 
have enough space, so we need to merge those paragraphs into one huge paragraph. 
As a consequence, it might become unclear about who is speaking, so please denote
the speaker when necessary.

To save space, we can add additional information at the end of dialogue. For 
example, 
\begin{quote}
``There, there, Mrs.~Meade'' said the doctor, basking obviously in the praise.
\end{quote}

Then avoid small talks in your dialogue. They might be useful in novels, but not 
here. And also, make the dialogue concise and brief. This not only because we have 
little space, but can also improves the dialogue quality.

You can cut out goodbyes and hellos to save space as well as to speed up your 
pace, as long as your readers understand the situation well.

Finally, try to give each speaker a unique voice, which greatly improves the 
story.

\subsection{Character Development}
We are probably not able to develop a character fully in \DHXX , but I will 
nevertheless introduce it here briefly, as a reason of its importance in 
literature.

The character development is not just to show a imaginary person to your readers, 
you need to show them how this person changes and transforms throughout the story. 
Is is the process of creating a person AS WELL AS the changes he goes through.

The characters are already given in the given story, but their personalities may 
not be decided yet. You can develop them on your own, but ensure that you have 
identified your characters and their roles. And when your try to develop them, 
think things in their situation --- What would you do if you were him?

Please refer to other resources for more information.

By now, our brief introduction about how to write a good \DHXX{} comes to an end.
But as Jesus said, ``The end is not yet.'' Our journey of learning never ends. And 
the process of learning English should be happening everyday.

Best wishes :) !